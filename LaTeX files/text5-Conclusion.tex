Creating compelling narratives from combinatorics is a devilish task. It has a long history that extends back beyond computational narratives to analog authors, such as Calvino, who was ``tempted by the diabolical idea of conjuring up all the stories that could be contained in a tarot deck" \cite{crossed_destinies}. And as the years have continued, the problem hasn't gotten easier to solve. Quite the contrary: our reach continues to exceed our grasp. Queneau, a fellow Oulipo member known for his combinatorial poetics, bemoaned the lack of computational assistance: ``we regret not having machines at our disposal: it is a continual lamento at our meetings" \cite[p.~322]{queneau_regret}. For us, the lucky denizens of the future, the continually increasing power of computers has provided ever-sharper tools to dig ourselves into even deeper trouble, in our search to plumb the fascinating depths of this medium.

In order to meet this challenge squarely, researchers and practitioners need a way to position themselves--not just as system designers creating thought-provoking prototypes for the sake of pure research--but as procedural content creators, fully engaging the intricacies of this peculiar problem as it extends into issues of authorial leverage.

However, as we have seen, this isn't something done lightly. Even with issues of aesthetics and audience aside, content production that isn't placeholder or ``stubbed in" requires real commitment and engagement with the system. And while yeoman's work, it is honest work that yields insights that might otherwise be hidden. This type of engagement is precisely the area this dissertation focalizes, over the author's six years conducting explorations in the field of procedural narrative.

They haven't been in vain. We offer one unified contribution, supported by three. First: a design methodology for evaluating the authorial leverage of dynamic narrative systems. This Authorial Leverage framework critically incorporates the intended media experience as a top-level consideration, a sort of ``bounding of the problem", that the subsequent parts of the framework guide the creator in addressing.

This framework wasn't created in a vacuum. It sprang from synthesized insights gathered from our second contribution: three novel systems for combinatoric narrative, two of which were taken all the way to finished media experiences.

Each of these three approaches applied combinatorial content production to a different set of systemic constraints, to explore how to leverage their different expressive affordances, with the ever-present goal of creating fully realized, cohesive narrative experiences. Because each system and work offers its own unique set of authoring challenges, we had to counteract this by making efforts to mitigate them--efforts formalized and abstracted into the Authorability Framework.

First, in Section \ref{sec:ib-experience-challenge} we tackled \textit{Ice-Bound}: a narrative system driven by thematic combinatorics. It used individual ``cards" of content triggered by pre-condition logic, whose thematic pools were driven by the interaction of the player as they engaged with the text, in the manner of ``sculptural fiction" (as coined by Reed in \cite{reed_changeful}). The (relatively) straightforward challenges of this work stemmed from the combinatorially large state space, a result of our desire to provide as high Explorability as possible. This large state space made authoring difficult due mostly to Clarity problems of not being able to surface which combinations lacked content. To tackle this, we came up with a series of tools, from simple pragmatic diagnostic tables to novel interactive state visualizations, which may be useful to other endeavors whose data has similar characteristics.

Next, in Section \ref{experience-challenge} we engaged choice combinatorics with StoryAssembler: a system that marries a forward state-space planner with an HTN-like planner, driven to recursively assemble fragments to satisfy (potentially dynamic) goals. These fragments presented a choice-driven narrative, which in our flagship work \textit{Emma's Journey} operated in dynamic lockstep with a mini-game generated by the Gemini system \cite{Gemini}. The challenges of this work were steep, born from challenges of choosing to provide an experience with a high amount of Contextuality between fragments, a high Complexity to author individual pieces of content, and a low amount of authorial Clarity on how newly-authored content would affect the content library as a whole. To combat this, we made some initial prototypes of an authoring tool to decrease Complexity, in addition to formalizing some system-specific authoring patterns. We also created a provisional narrative visualization tool to increase Clarity for content dynamics, though further work is to be done on this front, hopefully such that future potential users of the open-sourced StoryAssembler library may benefit.

Last, in Section \ref{sec:delve-experience-challenge} we turned our eyes to narrativization combinatorics and ontology-driven combinatorics in \textit{Delve}. Early approaches to generating character memories in \textit{Delve} sprang from ``bottom-up" approaches, seeking to establish first a system for generating simulation events, then narrativizing the combinatorics of their occurrence by filtering event logs via sifting patterns. Early experiments with this approach included LegendsWriter, which sought to apply this technique to Dwarf Fortress data to relate historical NPC stories. While initial progress was made, and the project can be formulated such that success is possible in even tiny incremental steps, much work remains to be done. Because of the challenges we saw would be posed in controlling the shape of character stories using an external simulation, we instead decided to choose a more controlled ``top-down" approach, such that we could more fully engage with the real aim of the project: ontology-driven combinatorics. In a way, \textit{Delve}'s system marries the two spaces of the previous systems in a unique form of interaction. The symbolic manipulation of 3D objects to translate into attribute grammar tags involves the same sorts of combinatorics we tackled in \textit{Ice-Bound}’s thematic system. The recursive, top-down generation of character memories calls to mind the same challenges encountered with StoryAssembler's fragments.

While work on authoring content for \textit{Delve} has yet to begin in earnest, this means it offers an excellent case study for using the Authorability Framework to plan for potential pain points, and formulate successful ways to incrementally make progress. Because it was so important to the intended experience that the player be able to manipulate memory objects in a variety of ways, and potentially replay the experience multiple times, we set the Explorability and Replayability to high, which sets us up for a steep authoring challenge that is only slightly mitigated by the potentially low Contextuality we can use, if need be. Areas to improve on, to increase our authorial leverage, are therefore tools that can reduce the high Complexity of authoring (by providing a more integrated location to author all the component pieces) and increase Clarity for the combinatoric space (so that we know when object affordances are sufficiently accounted for). We made an initial prototype visualization for memory surface text combinatorics, which proved useful, but more tooling will be needed to surface the dynamics of that space. In lieu of that, some potential avenues were explored for lowering the Contextuality, namely by leaning into the dream-like nature of the fiction in the subject matter of memories, and confining the tag combinatorics to things which were not plot-centric, but more like the garnishing details of \textit{Ice-Bound}'s shimmer text.

\section{Closing Thoughts}

Complex systems and a myriad of implementation details aside, some common themes emerged for the authoring tasks demanded by these works during their creation.

First: thinking about the type of experience you want to enable, even if the systemic work you are doing is abstract and theoretic, provides a convenient lens that can be used to further focus the directions of system implementation towards a compelling goal. By setting down the goals for Explorability, Replayability, and Contextuality, and thinking carefully about which ones are most critical to exhibit the strengths of the system you wish to develop (the ``system pillars") you are honing the edge of inquiry that can be used to triage features to cut or which must be pressed through despite difficulty, to best serve the research goal. Design work done here can head off potential problems before they become issues, by also shedding light on what types of authoring can lean into system strengths as much as possible.

Second: in terms of creating compelling media experiences, each system has a baseline of systemic complexity it offers, for a specific set of expressive affordances. We've spoken of the projects so far in terms of unilateral authoring complexity, but creative content is not a monolith! There's a ``sweet spot" for authoring with each system, and it's a consciously designed spot that leans into the system's strengths. The sooner that is figured out in the content creation process, the more authorial leverage can be maintained. But while every system has a baseline of complexity to offer, that Complexity can be pushed higher in controlled cases to get more generativity out of it, at the expense of decreasing authoring Clarity. There's a general concept in the games industry of ``hero" content--one risks more dynamism and interactivity in a tightly bounded manner, and spends the extra necessary time developing it, so that the player's model of the system complexity afterwards is higher than it might actually be. While that sort of stagecraft pragmatics is of limited use in academic areas, we can still use this in our experiences to gesture towards future capabilities, while not unilaterally committing ourselves to levels of complexity our authoring strategies can't currently support.

Third: every dynamic narrative system has some set of authoring patterns, which may have some generalizable attributes cross-system or between families of systems, but in general will be specific to the architecture and experience attempting to be provided. These authoring patterns can help increase authoring Clarity, by off-loading per-unit dynamics to more generalized ``set dynamics." Put another way: as content creation for the experience begins, identify repeated content dynamics authors gravitate towards, codify them, and make it easy to deploy them (through code templates or detailed authoring guidelines). This worked well for content authoring in \textit{Emma's Journey}, and may become more important to StoryAssembler as it gains exposure to more authors. As authoring begins in earnest for \textit{Delve}, such patterns will definitely be on the list of future developments.

To conclude: the exploration of combinatorial narrative systems is a fruitful and intriguing area for those seeking to develop novel media experiences. There is a surplus of opportunities to not only increase our knowledge of computational systems to support such experiences, but to plumb the new domains of design knowledge required by authoring in those systems as well. Each of these is important, and an integral part of the other. It is my hope that this framework and these narrative system explorations may prove valuable to others' endeavors, as we push new forms of the oldest art--that of storytelling--to new heights.